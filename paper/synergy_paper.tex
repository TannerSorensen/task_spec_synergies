\documentclass[preprint]{JASAnew}\usepackage[]{graphicx}\usepackage[]{color}
%% maxwidth is the original width if it is less than linewidth
%% otherwise use linewidth (to make sure the graphics do not exceed the margin)
\makeatletter
\def\maxwidth{ %
  \ifdim\Gin@nat@width>\linewidth
    \linewidth
  \else
    \Gin@nat@width
  \fi
}
\makeatother

\definecolor{fgcolor}{rgb}{0.345, 0.345, 0.345}
\newcommand{\hlnum}[1]{\textcolor[rgb]{0.686,0.059,0.569}{#1}}%
\newcommand{\hlstr}[1]{\textcolor[rgb]{0.192,0.494,0.8}{#1}}%
\newcommand{\hlcom}[1]{\textcolor[rgb]{0.678,0.584,0.686}{\textit{#1}}}%
\newcommand{\hlopt}[1]{\textcolor[rgb]{0,0,0}{#1}}%
\newcommand{\hlstd}[1]{\textcolor[rgb]{0.345,0.345,0.345}{#1}}%
\newcommand{\hlkwa}[1]{\textcolor[rgb]{0.161,0.373,0.58}{\textbf{#1}}}%
\newcommand{\hlkwb}[1]{\textcolor[rgb]{0.69,0.353,0.396}{#1}}%
\newcommand{\hlkwc}[1]{\textcolor[rgb]{0.333,0.667,0.333}{#1}}%
\newcommand{\hlkwd}[1]{\textcolor[rgb]{0.737,0.353,0.396}{\textbf{#1}}}%
\let\hlipl\hlkwb

\usepackage{framed}
\makeatletter
\newenvironment{kframe}{%
 \def\at@end@of@kframe{}%
 \ifinner\ifhmode%
  \def\at@end@of@kframe{\end{minipage}}%
  \begin{minipage}{\columnwidth}%
 \fi\fi%
 \def\FrameCommand##1{\hskip\@totalleftmargin \hskip-\fboxsep
 \colorbox{shadecolor}{##1}\hskip-\fboxsep
     % There is no \\@totalrightmargin, so:
     \hskip-\linewidth \hskip-\@totalleftmargin \hskip\columnwidth}%
 \MakeFramed {\advance\hsize-\width
   \@totalleftmargin\z@ \linewidth\hsize
   \@setminipage}}%
 {\par\unskip\endMakeFramed%
 \at@end@of@kframe}
\makeatother

\definecolor{shadecolor}{rgb}{.97, .97, .97}
\definecolor{messagecolor}{rgb}{0, 0, 0}
\definecolor{warningcolor}{rgb}{1, 0, 1}
\definecolor{errorcolor}{rgb}{1, 0, 0}
\newenvironment{knitrout}{}{} % an empty environment to be redefined in TeX

\usepackage{alltt}

\graphicspath{{./graphics/}}

% symbols
\usepackage[utf8]{inputenc}
\usepackage{siunitx}
\usepackage{tipa}
\usepackage{amsmath,mathtools,amssymb}

% URLs
\usepackage{url}

% table formatting
\usepackage{multirow,array}
\newcommand\Tstrut{\rule{0pt}{2.6ex}}         % = `top' strut
\newcommand\Bstrut{\rule[-0.9ex]{0pt}{0pt}}   % = `bottom' strut
\newcolumntype{L}[1]{>{\raggedright\let\newline\\\arraybackslash\hspace{0pt}}m{#1}}
\newcolumntype{C}[1]{>{\centering\let\newline\\\arraybackslash\hspace{0pt}}m{#1}}
\newcolumntype{R}[1]{>{\raggedleft\let\newline\\\arraybackslash\hspace{0pt}}m{#1}}
\usepackage{stackengine}

% overbrace on matrices
\newcommand\overmat[2]{%
  \makebox[0pt][l]{$\smash{\color{white}\overbrace{\phantom{%
    \begin{matrix}#2\end{matrix}}}^{\text{\color{black}#1}}}$}#2}
\newcommand\bovermat[2]{%
  \makebox[0pt][l]{$\smash{\overbrace{\phantom{%
    \begin{matrix}#2\end{matrix}}}^{#1}}$}#2}
\IfFileExists{upquote.sty}{\usepackage{upquote}}{}
\begin{document}



\title[Task dependence of articulator synergies]{Task-dependence of articulator synergies}

\thanks{Portions of this work were presented in 
``Factor analysis of vocal-tract outlines derived from real-time magnetic resonance imaging data,'' International Congress of Phonetic Sciences, Glasgow, UK, 2015,
``Characterizing vocal tract dynamics using real-time MRI,'' LabPhon 2016, Ithaca, NY, 2016,
``Characterizing vocal tract dynamics across speakers using real-time MRI,'' Proceedings of Interspeech, San Francisco, CA, 2016, 
``Decomposing vocal tract constrictions into articulator contributions using real-time MRI,'' Proceedings of the 7th International Conference on Speech Motor Control, Groningen, the Netherlands, 2017,
``Test-retest repeatability of articulatory strategies using real-time magnetic resonance imaging,'' Proceedings of Interspeech 2017, Stockholm, Sweden, 2017.}

\author{Tanner Sorensen}
\affiliation{Signal Analysis and Interpretation Laboratory, Ming Hsieh Department of Electrical Engineering, University of Southern California, Los Angeles, CA 90089, USA}
\affiliation{Department of Linguistics, University of Southern California, Los Angeles, CA 90089, USA}

\author{Asterios Toutios}
\affiliation{Signal Analysis and Interpretation Laboratory, Ming Hsieh Department of Electrical Engineering, University of Southern California, Los Angeles, CA 90089, USA}

\author{Louis Goldstein}
\affiliation{Department of Linguistics, University of Southern California, Los Angeles, CA 90089, USA}

\author{Shrikanth Narayanan}
\affiliation{Signal Analysis and Interpretation Laboratory, Ming Hsieh Department of Electrical Engineering, University of Southern California, Los Angeles, CA 90089, USA}
\affiliation{Department of Linguistics, University of Southern California, Los Angeles, CA 90089, USA}

\begin{abstract}
In speech production, the motor system organizes articulators such as the jaw, tongue, and lips into synergies whose function is to produce speech sounds by forming constrictions at the phonetic places of articulation.
%
The present study tested whether synergies with different places of articulation differed in terms of inter-articulator coordination.
%
The test was conducted using a real-time magnetic resonance imaging biomarker that was computed using a computational model of the direct and differential kinematics of the vocal tract. 
%
The present study is the first to estimate the direct and differential kinematics of the vocal tract from speech production data.
%
The study determined that the jaw contributes more to anterior constrictions at the alveolar place of articulation than to posterior constrictions at the velar and pharyngeal places of articulation, and that there was substantial inter-speaker variability in jaw, tongue, and lip usage at the bilabial and palatal places of articulation. 
%
Differences in inter-articulator coordination at different phonetic places of articulation support the claim that articulator synergies are task-dependent.
\end{abstract}


\maketitle


\section{Introduction}

An articulator synergy is a functional grouping of articulators such as the jaw, tongue, and lips whose coordinated movements produce constrictions during speech~\cite{turvey1977preliminaries}. 
%
Phonetic studies have shown that the coordination of articulators differs depending on where in the vocal tract the synergy produces a constriction. 
%
For instance, mechanical perturbations to jaw position during a bilabial constriction induced compensatory lip movement with no compensatory tongue movement, 
%
whereas mechanical perturbations to jaw position during a lingual constriction induced compensatory tongue movement with no compensatory lip movement~\citep{kelso1984functionally}. 
%
Our previous study on unperturbed speech suggested that healthy adult speakers of American English may use the jaw more for anterior constrictions at the bilabial, alveolar, and palatal places of articulation than for posterior constrictions at the velar and pharyngeal places of articulation~\citep{Sorensen+2016}. 
%
Differences in inter-articulator coordination at different places of articulation support the \textit{task-dependence} of articulator synergies~\citep{latash2008synergy}. 

The present study investigated the task-dependence of articulator synergies by quantifying the percent contribution of each articulator to narrowing or widening the vocal tract at the synergy's place of articulation. 
%
In the Task Dynamics model of speech production~\citep{saltzman1989dynamical}, the percent contribution of each articulator in a synergy is determined by assigning weights to the articulators. 
%
Whereas some studies manually assigned weights to the articulators based on theoretical considerations~\citep[for example, see][for an assignment of weights based on articulator mass]{simko2010embodied}, the present study is the first to obtain a quantitative readout of these weights from speech production data. 

Advances in magnetic resonance imaging (MRI) have achieved a balance among the competing factors of temporal resolution, spatial resolution, and signal-to-noise ratio that provides a rich source of speech production data for the present study~\citep{scott2014speech}. Real-time MRI pulse sequences and reconstruction techniques allow for the capture and visualization of the motion of the jaw, tongue, and lips with \SI{12}{\milli\second} temporal resolution~\citep{toutios2016advances,lingala2016state}. 
%
The present study used real-time MRI to quantify articulator synergies in terms of the percent contribution of each articulator to constriction tasks at the phonetic places of articulation.
%
The proposed measure of articulator synergies is derived from in vivo MRI as a descriptor of articulator synergies. This measure meets the definition of quantitative imaging biomarker as ``an objective characteristic derived from an in vivo image measured on a ratio or interval scale as an indicator of normal biological processes''~\citep{kessler2015emerging,sullivan2015metrology}. 





The algorithm for computing the articulator synergy biomarker involves a computational model of the direct and differential kinematics of the vocal tract~\citep{lammert2013statistical}, based on the Task Dynamics model of speech production~\citep{saltzman1989dynamical}. 
%
The direct kinematics relates the position and shape of articulators to the corresponding degree of constriction (measured in millimeters; hereafter, constriction degree) at the phonetic places of articulation by means of a function called the forward kinematic map. 
%
The differential kinematics relates small increments of articulator movement to the resulting changes in the constriction degrees by means of the jacobian matrix of the forward kinematic map. 
%
The algorithm computes the percent contribution of each articulator to narrowing or widening the vocal tract at the synergy's place of articulation using the jacobian matrix of the forward kinematic map. 






The research goals of the present study are 
%
(i) to estimate and evaluate the direct and differential kinematics of the vocal tract from MRI, 
%
(ii) to design and evaluate a biomarker of articulator synergies based on the estimated kinematics, and
%
(iii) to use the articulator synergy biomarker to test the task-dependence of articulator synergies by determining whether the jaw contributes more for anterior constrictions at the bilabial, alveolar, and palatal places of articulation than for posterior constrictions at the velar and pharyngeal places of articulation. 
%
The study demonstrated low error in the estimator of the direct and differential kinematics and consistent estimation of the articulator synergy biomarker in a test-retest repeatability experiment.
%
A comparison of the parameters of a mixed effects model fitted to the biomarker values demonstrated that speakers use the jaw more to produce constrictions at the alveolar place of articulation than to produce constrictions at the velar and pharyngeal places of articulation. 
%
The pattern of jaw, tongue, and lip usage was heterogeneous for the bilabial and palatal places of articulation.
%
The implication of the results is that the articulator synergy biomarker has adequate precision to characterize healthy variability in articulator synergies, and that articulator synergies are task-dependent in that they have different patterns of inter-articulator coordination depending on their place of articulation. 


The paper is organized as follows.
%
Section~\ref{sec:mri} describes the MRI experiment, scanner sequence, participant characteristics, and method for manually annotating the start and end time-points of utterances in the real-time MRI.
%
Sections~\ref{sec:cd} and~\ref{sec:gfa} estimate the constriction degrees and parameters of articulator shape and position, respectively, that are related by the direct and differential kinematics. 
% 
Section~\ref{sec:forwardkinematicmap} estimates the direct and differential kinematics and evaluates the model through cross-validation. 
%
Section~\ref{sec:articulator synergy biomarker} defines the articulator synergy biomarker and evaluates the precision of the articulator synergy biomarker in a test-retest experiment, which estimated the repeatability of the articulator synergy biomarker in two experiments on the same day under the same experimental conditions. 
% 
Section~\ref{sec:taskspec} uses the articulator synergy biomarker to test the task-dependence of articulator synergies.   



\section{Magnetic resonance imaging}
\label{sec:mri}

\subsection{Experiment protocol}

Each participant took part in one session. The study personnel explained the nature of the experiment and the protocol to the participant before each scan. The participant lay on the scanner table in a supine position. The head was fixed in place by foam pads inserted between the temple and the receiver coil on the left and right sides of the head. The participant read visually presented text from a paper card taped to the scanner bore in front of their face. The speech corpus included real-time MRI videos of the isolated vowel-consonant-vowel utterances \textipa{[apa], [ata], [aka], [aja]}. The participant produced each vowel-consonant-vowel utterance ten times. The study personnel removed the participant from the scanner for a short break, and then repeated the experiment. After completing the session, the speaker was paid for their participation in the study. The USC Institutional Review Board approved the data collection procedures. 




\subsection{Imaging parameters}

Data were acquired on a Signa Excite HD \SI{1.5}{\tesla} scanner (General Electric Healthcare, Waukesha WI) with gradients capable of \SI[per-mode=symbol]{40}{\milli\tesla\per\meter} amplitude and \SI[per-mode=repeated-symbol]{150}{\milli\tesla\per\meter\per\milli\second} slew rate. A custom eight-channel upper airway coil was used for radio frequency signal reception. The coil has two 4-channel arrays. 
%
A real-time MRI pulse sequence based on a spiral fast gradient echo pulse sequence was used. 
%
The real-time MRI pulse sequence parameters were the following: 
%
\SI{200 x 200}{\milli\meter} field of view, 
\SI{2.4 x 2.4}{\milli\meter} reconstructed in-plane spatial resolution, 
\SI{6}{\milli\meter} slice thickness,
\SI{6}{\milli\second} TR,
\SI{3.6}{\milli\second} TE,
\SI{15}{\degree} flip angle,
\num{13} spiral interleaves for full sampling.
%
The scan plane was manually aligned to the head. 
%
Images were retrospectively reconstructed to a temporal resolution of \SI{12}{\milli\second} (\SI{6}{\milli\second} TR times two spirals per image, 83 frames per second). 
%
Reconstruction was performed using the Berkeley Advanced Reconstruction Toolbox~\citep{uecker2015berkeley}.






\subsection{Participant characteristics}

The data-set included eight (4 male, 4 female) speakers of American English. Five participants were native speakers of American English. None of the participants reported speech pathology or abnormal hearing. Table~\ref{tab:subj2} lists age and state of origin. 

\begin{table}
\centering
\begin{tabular}{l l l L{0.2\linewidth} l}
\hline
ID & age & gender & state of origin & native language \Tstrut \Bstrut \\
\hline
F1 & 25 & F & Rhode Island & American English \Tstrut \\
F2 & 28 & F & Texas & American English \\
F3 & 24 & F & Nebraska & American English \\
F4 & 29 & F & Korea & Korean \\
M1 & 29 & M & Iowa & American English \\
M2 & 27 & M & United Arab Emirates & American English \\
M3 & 26 & M & Germany & German \\
M4 & 39 & M & Greece & Greek \\
\hline
& \shortstack[l]{\Tstrut median: 28 \\ range: 24--39}
& \shortstack[l]{\Tstrut 4 male \\ 4 female}
& 
& \shortstack[l]{5 American English \\ 3 other}\Tstrut \Bstrut \\
\hline 
\end{tabular}
\caption{Participant characteristics of the test-retest data-set}
\label{tab:subj2}
\end{table}






\subsection{Time-point annotation}
\label{subsec:timepointannotation}

Vocal tract constrictions were manually identified in the real-time MRI videos. 
%
The video frames were inspected on a computer monitor. 
%
Guided by graphical presentation of real-time MRI video frames and auditory presentation of a denoised speech audio signal recorded in the scanner bore, the study personnel manually identified the intervals of time during which the vocal tract produced oral constrictions for the consonants of the vowel-consonant-vowel sequences \textipa{[apa]}, \textipa{[ata]}, \textipa{[aka]}, and \textipa{[aja]}. 
%
Study personnel annotated the frame number of the first and last frames in which there was visible movement of the lips (for \textipa{[apa]}) or tongue (for \textipa{[ata]}, \textipa{[aka]}, and \textipa{[aja]}). 





\section{Constriction degree measurement}
\label{sec:cd}

The contours of articulators were identified in the real-time MRI videos and tracked automatically during vocal tract constrictions~\citep{bresch2009region}. The algorithm was manually initialized with templates matching vocal tract contours during the sounds \textipa{[a], [i], [p], [t], [k]} (Fig.~\ref{fig:segtemp}). 
%
If visual inspection revealed clear errors, then the algorithm initialization was manually corrected and the contours were re-submitted to the algorithm. This was repeated as needed until no clear contour tracking errors remained. 
%
See Fig.~\ref{fig:segtemp} for example segmentation results.

\begin{figure}

\includegraphics[width=\linewidth]{SegTempFigure.pdf}

\caption{(color online) {\bf a.} Study personnel manually created subject-specific templates for \textipa{[a]}, \textipa{[i]}, \textipa{[p]}, \textipa{[t]}, \textipa{[k]}, which were then automatically deformed to fit the articulator contours in the real-time magnetic resonance images. The different articulator contours are in different colors.
{\bf b.} Articulator contour segmentation of a sequence of real-time magnetic resonance images acquired in the transition from vowel \textipa{[a]} to glide \textipa{[j]} in \textipa{[aja]}. Reconstructed frame rate was downsampled for presentation.}
\label{fig:segtemp}
\end{figure}



Constrictions were quantified by measuring change in constriction degree as a local descriptor of airway shape at a phonetic place of articulation. 
%
The constriction degree was the distance between opposing structures at the place of articulation. 
%
The opposing structures were the upper and lower lips for [p], tongue and alveolar place for [t], tongue and palatal place for [j], tongue and soft palate for [k], and tongue and rear pharyngeal wall for [a]. 
%
The contour tracking algorithm automatically identified the upper lip, lower lip, tongue, hard palate, soft palate, and rear pharyngeal wall~\citep{bresch2009region}.  
%
The anterior \num{1/4} of the hard palate was the alveolar place of articulation. 
%
The posterior \num{1/2} of the hard palate was the palatal place of articulation.
%
The velar place was bounded anteriorly by the anterior edge of the soft palate and extended posteriorly over \num{1/8} of the total soft palate contour, which included the oral, uvular, oropharyngeal, and nasal surfaces of the soft palate (cf. brown soft palate contours in~\ref{fig:segtemp}).
%
The pharyngeal place was the posterior pharyngeal wall, bounded superiorly by the velopharyngeal port and inferiorly by the larynx. 


An algorithm automatically measured the degree of constriction at the phonetic places of articulation in each video frame. Bilabial constriction degree was the minimum distance between the upper lip and lower lip. Constriction degrees in the oral cavity were the minimum distances from the tongue to the alveolar, palatal, and velar place. The constriction degree in the pharynx was the minimum distance from the base of tongue to the posterior pharyngeal wall. 
%
Fig.~\ref{fig:constrictions} illustrates the constriction degree measurements at the phonetic places of articulation.


\begin{figure}

\includegraphics[width=\linewidth]{ConstrictionsFigure.pdf}

\caption{(color online) Constriction degrees at the phonetic places of articulation in the transition from vowel \textipa{[a]} to glide \textipa{[j]} in the sequence \textipa{[aja]}. The phonetic places of articulation (blue lines) are bilabial place, alveolar place, palatal place, velar place, velopharyngeal port, and pharyngeal place. Reconstructed frame rate was downsampled for presentation.}
\label{fig:constrictions}
\end{figure}




\section{Guided factor analysis of vocal tract shapes}
\label{sec:gfa}




\subsection{Objective of the guided factor analysis}
\label{subsec:objectivesoftheguidedfactoranalysis}

The objective of the guided factor analysis was to parameterize the vocal tract contours $\mathbf{X} \in \mathbb{R}^{n\times p}$, where $n$ is the number of images and $p$ is the number of contour vertices, as the linear combination of factors $\mathbf{f}_1, \mathbf{f}_2, \ldots, \mathbf{f}_q \in \mathbb{R}^p$ such that each factor characterizes spatial variation in the position and shape of an articulator. 
%
The coefficients $\mathbf{w}_{\cdot,1},\mathbf{w}_{\cdot,2},\ldots,\mathbf{w}_{\cdot,p} \in \mathbb{R}^n$ of the linear combination are factor scores that characterize temporal variation in the position and shape of the articulators. 
%
The guided factor analysis is based on the approach of~\citet{toutios2015factor}.


Factors $\mathbf{f}_1, \mathbf{f}_2, \ldots, \mathbf{f}_q \in \mathbb{R}^p$ are the columns of the matrix $\mathbf{F} \in \mathbb{R}^{p\times q}$. 
%
Rows $\mathbf{w}_{1,\cdot},\mathbf{w}_{2,\cdot},\ldots,\mathbf{w}_{n,\cdot}$ of the matrix $\mathbf{W} \in \mathbb{R}^{n\times q}$ contain the factor scores for images $1,2,\ldots,n$ of the real-time MRI dataset. 
%
Thus, the contour vertices $\mathbf{x}_{i,\cdot}$ for the $i$th image of the real-time MRI dataset are approximately equal to the following linear combination of the factors. 
%
\begin{align}
\label{eq:linearcombo}
\mathbf{x}_{i,\cdot}
 &=
  \mathbf{w}_{i,\cdot} \mathbf{F}^\intercal \\
 &=
  w_{i,1} \mathbf{f}_{\cdot,1}^\intercal
  + w_{i,2} \mathbf{f}_{\cdot,2}^\intercal
  + \ldots
  + w_{i,q} \mathbf{f}_{\cdot,q}^\intercal
\end{align}




The motion of the tongue and lips systematically co-occurs with motion of the jaw due to mechanical constraints and regularities in motor commands. For this reason, we seek a factor that corresponds to the motion of the jaw along with the concomitant motion of the tongue and lips.


The set of $q_\text{jaw}$ jaw factors $\mathbf{F}_\text{jaw} \in \mathbb{R}^{p\times q_\text{jaw}}$ parameterizes motion of the jaw along with concomitant motion of the tongue, lips, and velum.
%
The set of $q_\text{tongue}$ tongue factors $\mathbf{F}_\text{tongue} \in \mathbb{R}^{p\times q_\text{tongue}}$ parameterizes motion of the tongue that is independent of the jaw. 
%
The set of $q_\text{lips}$ lip factors $\mathbf{F}_\text{lips} \in \mathbb{R}^{p\times q_\text{lips}}$ parameterizes motion of the lips that is independent of the jaw.
%
The set of $q_\text{velum}$ velum factors $\mathbf{F}_\text{velum} \in \mathbb{R}^{p\times q_\text{velum}}$ parameterizes motion of the velum.
%
The full set of factors can be written as the block matrix $\mathbf{F} \in \mathbb{R}^{p\times q}$, where $q=q_\text{jaw}+q_\text{tongue}+q_\text{lips}+q_\text{velum}$. 
% 
\begin{equation}
\label{eq:Fblock}
\mathbf{F} = 
\left[
\begin{array}{c|c|c|c}
\mathbf{F}_\text{jaw} 
& \mathbf{F}_\text{tongue}
& \mathbf{F}_\text{lips}
& \mathbf{F}_\text{velum}
\end{array}
\right]
\end{equation}
%
The corresponding set of factor scores can be written as the block matrix $\mathbf{W} \in \mathbb{R}^{p\times q}$, where $q=q_\text{jaw}+q_\text{tongue}+q_\text{lips}+q_\text{velum}$. 
% 
\begin{equation}
\label{eq:Wblock}
\mathbf{W} = 
\left[
\begin{array}{c|c|c|c}
\mathbf{W}_\text{jaw} 
& \mathbf{W}_\text{tongue}
& \mathbf{W}_\text{lips}
& \mathbf{W}_\text{velum}
\end{array}
\right]
\end{equation}

Section~\ref{subsec:preliminaries} is a preliminary technical note. Section~\ref{subsec:jawfactors} derives the jaw factors. Section~\ref{subsec:residfactors} derives the tongue, lip, and velum factors. Section~\ref{subsec:weights} derives the factor scores. 





\subsection{Preliminaries to the guided factor analysis}
\label{subsec:preliminaries}

Different steps of the guided factor analysis focus on different articulators.
%
The guided factor analysis uses a projection operator to set to zero the contour vertices of articulators not under analysis in a given step of the analysis.
% 
For example, in order to derive the matrix $\mathbf{X}_\text{jaw} \in \mathbb{R}^{n\times p}$ that contains only jaw contour vertices, the guided factor analysis sets to zero the contour vertices (i.e., columns of $\mathbf{X}$) of all non-jaw articulators and leaves the contour vertices of the jaw unchanged. 
%
Specifically, the non-jaw contour vertices are set to zero by multiplying $\mathbf{X}$  
% 
by the diagonal projection matrix $\mathbf{P}_\text{jaw} \in \mathbb{R}^{p\times p}$. 
% 
We have that $p_{i,i}=1$ if the $i$th column of $\mathbf{X}$ is a jaw vertex. Otherwise, $p_{i,i}=0$. 
% 
This projection operator sets to zero the columns of $\mathbf{X}$ corresponding to non-jaw contour vertices. 
% 
If jaw contour vertices are in columns $q_1,q_2,\ldots,q_\ell$ of $\mathbf{X}$, then the projection works out to the following. 
% 
\begin{equation}
\begin{split}
  \mathbf{X}_\text{jaw} &= \mathbf{X} \mathbf{P}_\text{jaw} \\[10pt]
%
  \left[ {\begin{array}{ccccc}
   0 & \bovermat{\text{jaw contour vertices}}{x_{1,q_1} &  \cdots & x_{1,q_\ell}} & 0 \\
   0 & x_{2,q_1} &  \cdots & x_{2,q_\ell} & 0 \\
   \vdots & \vdots & & \vdots & \vdots \\
   0 & x_{m,q_1} & \cdots & x_{m,q_\ell} & 0 \\
  \end{array} } \right]
%
  &= 
%
   \mathbf{X}
%
   \left[ {\begin{array}{ccccc}
   \bovermat{\text{columns } q_1,\ldots,q_\ell}{0&&&&}\\
   & 1 &&&\\
   && \ddots &&\\
   &&& 1 &\\
   &&&& 0 \\
  \end{array} } \right]
\end{split}
\end{equation}
%
Similarly, the matrices $\mathbf{X}_\text{tongue},\mathbf{X}_\text{lips},\mathbf{X}_\text{velum} \in \mathbb{R}^{n\times p}$ focus on the tongue and lip contours, respectively. 
% 
Summing such matrices produces a matrix that corresponds to a set of articulators. For instance, the matrix $\mathbf{X}_{\text{jaw,tongue,lips}} = \mathbf{X}_\text{jaw} + \mathbf{X}_\text{tongue} + \mathbf{X}_\text{lips}$ corresponds to the jaw, tongue, and lips.





\subsection{Jaw factors}
\label{subsec:jawfactors}

We first obtained the factors $\mathbf{F}_\text{jaw}$ that capture the contribution of the jaw to vocal tract shaping (see the jaw factor in Fig.~\ref{fig:gfa}). 
% 
We performed principal component analysis of the jaw contour vertices through eigendecomposition of the covariance matrix $\boldsymbol{\Sigma}_\text{jaw} = \mathbf{X}_\text{jaw}^\intercal \mathbf{X}_\text{jaw}/(n-1)$ into an orthogonal matrix $\mathbf{Q}_\text{jaw} \in \mathbb{R}^{p\times q_\text{jaw}}$ whose columns are the principal axes of $\mathbf{X}_\text{jaw}$ and a diagonal matrix $\boldsymbol{\Lambda}_\text{jaw} \in \mathbb{R}^{q_\text{jaw} \times q_\text{jaw}}$ whose diagonal entries are the variances of $\mathbf{X}_\text{jaw}$ on the principal axes.
% 
\begin{equation}
\boldsymbol{\Sigma}_\text{jaw} = \mathbf{Q}_\text{jaw}\boldsymbol{\Lambda}_\text{jaw} \mathbf{Q}_\text{jaw}^{-1}
\end{equation}
%
The principal axes $(\mathbf{q}_\text{jaw})_{\cdot,1}, (\mathbf{q}_\text{jaw})_{\cdot,2}, \ldots, (\mathbf{q}_\text{jaw})_{\cdot,q_\text{jaw}}$ and the variances $(\lambda_\text{jaw})_{1,1}, (\lambda_\text{jaw})_{2,2}, \ldots, (\lambda_\text{jaw})_{q_\text{jaw},q_\text{jaw}}$ on these axes capture the direction and variance of jaw motion but do not capture the concomitant motion of the tongue and lips 
% 
because each principal axis has zero length in the direction of non-jaw articulators 
% 
(e.g., $(\mathbf{q}_\text{jaw})_{\cdot,1}^\intercal \mathbf{P}_\text{tongue} = \mathbf{0}^\intercal$ and $(\mathbf{q}_\text{jaw})_{\cdot,1}^\intercal \mathbf{P}_\text{lips} = \mathbf{0}^\intercal$). 
% 
Rather than use the variance on the principal axes to characterize jaw motion, we instead use factors that capture jaw motion along with concomitant tongue and lip motion. 
% 
The factors $(\mathbf{f}_\text{jaw})_{\cdot,1},(\mathbf{f}_\text{jaw})_{\cdot,2},\ldots,(\mathbf{f}_\text{jaw})_{\cdot,q_\text{jaw}}$ are the columns of matrix $\mathbf{F}_\text{jaw} \in \mathbb{R}^{p\times q_\text{jaw}}$. 
%
\begin{equation}
\mathbf{F}_\text{jaw}
 = \boldsymbol{\Sigma}_\text{jaw,tongue,lips} \mathbf{Q}_\text{jaw} \boldsymbol{\Lambda}_\text{jaw}^{-1/2}
\end{equation}
%
Column $(\mathbf{f}_\text{jaw})_{\cdot,i}$ is the vector of covariances between the jaw, tongue, and lip contour vertices and the z-scored component scores $\mathbf{X} (\mathbf{q}_\text{jaw})_{\cdot,i} / \sqrt{(\lambda_\text{jaw})_{i,i}}$ for the $i$th jaw principal component. 
%
Thus, the factors capture motion of the tongue and lips which accompanies the motion of the jaw. 
%
Note that the matrix $\boldsymbol{\Lambda}_\text{jaw}^{-1/2}$ is the inverse of the principal square root of $\boldsymbol{\Lambda}_\text{jaw}$.
%
Postmultiplying $\boldsymbol{\Sigma}_\text{jaw,tongue,lips} \mathbf{Q}_\text{jaw}$ by $\boldsymbol{\Lambda}_\text{jaw}^{-1/2}$ z-scores the component scores $\mathbf{X}\mathbf{Q}_\text{jaw}$, whose covariances with the jaw, tongue, and lip contour vertices $\mathbf{X}_\text{jaw,tongue,lips}$ are the entries of the jaw factors $\mathbf{F}_\text{jaw}$.

The column space $\mathrm{Col}(\mathbf{F}_\text{jaw})$ of $\mathbf{F}_\text{jaw}$ is a $q_\text{jaw}$-dimensional subspace of $\mathbb{R}^p$. 
%
Variance within $\mathrm{Col}(\mathbf{F}_\text{jaw})$ captures jaw contour motion and concomitant tongue and lip motion. 
%
The projection $\mathbf{\hat{X}}$ of the data $\mathbf{X}$ on the space $\mathrm{Col}(\mathbf{F}_\text{jaw})$ is obtained through the Moore-Penrose pseudoinverse $\mathbf{F}_\text{jaw}^+$ of $\mathbf{F}_\text{jaw}$. 
%
\begin{equation} \label{eq:XXuu}
\mathbf{\hat{X}} 
= \mathbf{X} 
  \mathbf{F}_\text{jaw}
  \mathbf{F}_\text{jaw}^+ 
\end{equation}
%
The null space $\mathrm{Null}(\mathbf{F}_\text{jaw})$ is a $(p-q_\text{jaw})$-dimensional subspace of $\mathbb{R}^p$. 
% 
Variance within $\mathrm{Null}(\mathbf{F}_\text{jaw})$ captures velum motion along with the part of tongue and lip motion that is independent of jaw motion. 
%
Section~\ref{subsec:residfactors} describes the factors that characterize the variance of the tongue, lips, and velum within $\mathrm{Null} \left( \mathbf{F}_\text{jaw} \right)$. 





\subsection{Tongue, lip, and velum factors}
\label{subsec:residfactors}

This section describes the procedure for obtaining the factors $\mathbf{F}_\text{tongue}$ that capture the contribution of the tongue to vocal tract shaping  (see the tongue factors in Fig.~\ref{fig:gfa}). 
% 
The projection $\mathbf{\tilde{X}}$ of the data matrix $\mathbf{X}$ on the space $\mathrm{Null}(\mathbf{F}_\text{jaw})$ is the contour vertex motion that is independent of the jaw.
%
\begin{align} \label{eq:XXX}
\mathbf{\tilde{X}} 
&= \mathbf{X} \left( \mathbf{I}_p - \mathbf{F}_\text{jaw}\mathbf{F}_\text{jaw}^+ \right) \\
&= \mathbf{X} - \mathbf{\hat{X}} 
\end{align}
%
Specifically, $\mathbf{\tilde{X}}$ is independent of the jaw in the sense that it is statistically independent of $\mathbf{\hat{X}}$. 
%
% Appendix~\ref{sec:independence} provides a proof of independence.  





We performed principal component analysis of the tongue contour vertices through eigendecomposition of the covariance matrix $\boldsymbol{\tilde{\Sigma}}_\text{tongue} = \mathbf{\tilde{X}}_\text{tongue}^\intercal \mathbf{\tilde{X}}_\text{tongue}/(n-1)$ into an orthogonal matrix $\mathbf{Q}_\text{tongue}$ whose columns are the principal axes of $\mathbf{\tilde{X}}_\text{tongue}$ and a diagonal matrix $\boldsymbol{\Lambda}_\text{tongue}$ whose diagonal entries are the variances of $\mathbf{\tilde{X}}_\text{tongue}$ on the principal axes.
% 
\begin{equation}
\boldsymbol{\tilde{\Sigma}}_\text{tongue} = \mathbf{Q}_\text{tongue} \boldsymbol{\Lambda}_\text{tongue} \mathbf{Q}_\text{tongue}^{-1}
\end{equation}
%
The principal axes $(\mathbf{q}_\text{tongue})_{\cdot,1}$, $(\mathbf{q}_\text{tongue})_{\cdot,2}, \ldots$, $(\mathbf{q}_\text{tongue})_{\cdot,q_\text{tongue}}$ and the variances $(\lambda_\text{tongue})_{1,1}$, $(\lambda_\text{tongue})_{2,2}, \ldots$, $(\lambda_\text{tongue})_{q_\text{tongue},q_\text{tongue}}$ on these axes capture the direction and variance of tongue motion.


Column $(\mathbf{f}_\text{tongue})_{\cdot,i}$ of the tongue factor matrix $\mathbf{F}_\text{tongue}$ is the vector of covariances between the tongue contour vertices and the z-scored component scores $\mathbf{X} (\mathbf{q}_\text{tongue})_{\cdot,i} / \sqrt{(\lambda_\text{tongue})_{i,i}}$ for the $i$th tongue principal component. 
%
\begin{equation}
\mathbf{F}_\text{tongue}
 = \boldsymbol{\tilde{\Sigma}}_\text{tongue} \mathbf{Q}_\text{tongue} \boldsymbol{\Lambda}_\text{tongue}^{-1/2}
\end{equation}


The column space $\mathrm{Col}(\mathbf{F}_\text{tongue})$ of $\mathbf{F}_\text{tongue}$ is a $q_\text{tongue}$-dimensional subspace of $\mathbb{R}^p$. 
%
Variance within $\mathrm{Col}(\mathbf{F}_\text{tongue})$ captures tongue contour motion that is not concomitant with jaw motion. 
%
The procedure for deriving lip and velum factors is the same as for the tongue factors except that $\mathbf{X}_\text{lips}$ or $\mathbf{X}_\text{velum}$ is substituted for $\mathbf{X}_\text{tongue}$.




\subsection{Factor scores}
\label{subsec:weights}

According to Equation~\ref{eq:linearcombo}, the data matrix $\mathbf{X}$ is parameterized as the matrix product $\mathbf{W}\mathbf{F}^\intercal$. 
%
Sections~\ref{subsec:jawfactors} and~\ref{subsec:residfactors} specified the factors $\mathbf{F}$. 
%
This section derives the factor scores $\mathbf{W}$ from the factors $\mathbf{F}$ and the data matrix $\mathbf{X}$. 
%
\begin{equation}
\mathbf{W} 
 = \mathbf{X} \left( \mathbf{F}^\intercal \right) ^+
\end{equation}
%
In image $i$ of the real-time MRI dataset, the vocal tract contours $\mathbf{x}_{i,\cdot}$ is approximately equal to the linear combination $w_{i,1} \mathbf{f}_1^\intercal + w_{i,2} \mathbf{f}_2^\intercal + \ldots + w_{i,q} \mathbf{f}_q^\intercal$. The factor scores $w_{i,\cdot }$ are the coefficients of the linear combination. Variance of the factor scores parameterizes the temporal variability of vocal tract shape. 

\begin{figure}

\includegraphics[width=0.99\linewidth]{FactorsFigure.pdf}

\caption{(color online) Factors obtained for one participant with one jaw factor, four tongue factors, two lip factors, and one velum factor ($q_\text{jaw} = 1$, $q_\text{tongue} = 4$, $q_\text{lip} = 2$, $q_\text{velum} = 1$). 
Each factor characterizes spatial variation in the position and shape of an articulator. 
The jaw factor captures jaw contour motion and concomitant tongue and lip motion.
The tongue, lip, and velum factors capture tongue and lip contour motion that is not concomitant with jaw motion. 
The articulator contours in a given real-time magnetic resonance image are parameterized as the linear combinations of the factors. 
For a given participant, the coefficients of this linear combination change from image to image as the articulators move, while the factors remain constant.}
\label{fig:gfa}
\end{figure}







\section{Direct and differential kinematics}
\label{sec:forwardkinematicmap}

\subsection{Estimation of the direct and differential kinematics}

The forward kinematic map is the function that maps the shape and position of the articulators (here parameterized by factor scores) to the corresponding constriction degrees at the phonetic places of articulation~\citep{lammert2013statistical}. 
%
Although the function is nonlinear, in the neighborhood of a given point $\mathbf{w}_{j,\cdot}$ the function can be linearized to have the form of a linear system of equations.
%
Specifically, the linearized forward kinematic map is a function of the following form, where $\mathbf{w} \in \mathbb{R}^q$ is a vector of factor scores and $\mathbf{z} \in \mathbb{R}^{m}$ is a vector of constriction degrees at the $m$ phonetic places of articulation.
%
\begin{align}
\mathbf{z} 
&= 
\mathbf{G}\left( \mathbf{w} \right) 
\left[ \begin{array}{c} 1 \\ \mathbf{w} \end{array} \right] \\
&= 
\left[ \begin{array}{cc} 
\boldsymbol{\mu}\left(\mathbf{w}\right) & \mathbf{J}\left(\mathbf{w}\right) 
\end{array} \right]
\left[ \begin{array}{c} 1 \\ \mathbf{w} \end{array} \right]
\end{align}
%
The first column $\mathbf{g}_{\cdot,1}(\mathbf{w})$ of the matrix $\mathbf{G}(\mathbf{w}) \in \mathbb{R}^{m\times (q+1)}$ is the vector $\boldsymbol{\mu}(\mathbf{w}) \in \mathbb{R}^m$ of intercepts for each constriction degree in the neighborhood of $\mathbf{w}$. 
%
The remaining columns $\mathbf{g}_{\cdot,2}(\mathbf{w}), \ldots, \mathbf{g}_{\cdot,q+1}(\mathbf{w})$ define the jacobian $\mathbf{J}(\mathbf{w})$ of the forward kinematic map in the neighborhood of $\mathbf{w}$. 
% 
\begin{align}
\mathbf{J}(\mathbf{w}) 
&=
\left[ \begin{array}{ccc} 
\frac{\partial \mathbf{z}_1}{\partial \mathbf{w}_1} & \cdots & \frac{\partial \mathbf{z}_1}{\partial \mathbf{w}_q} \\
\vdots & \ddots & \vdots \\
\frac{\partial \mathbf{z}_m}{\partial \mathbf{w}_1} & \cdots & \frac{\partial \mathbf{z}_m}{\partial \mathbf{w}_q} \\
\end{array} \right]
\end{align}
%
The forward kinematic map is estimated using weighted least squares. 
% 
The estimator $\mathbf{\hat{g}}_{i,\cdot}$ of row $\mathbf{g}_{i,\cdot}$ of $\mathbf{G}$ is defined locally at point $\mathbf{w}$ as the function that minimizes the weighted sum of squared errors
%
\begin{equation}
\mathrm{SSE} 
=
\left( \mathbf{z}_{\cdot,i} - \mathbf{\hat{z}}_{\cdot,i} \right)^\intercal
\mathbf{C}(\mathbf{w})
\left( \mathbf{z}_{\cdot,i} - \mathbf{\hat{z}}_{\cdot,i} \right)
\end{equation}
%
for $i=1,2,\ldots,m$,
% 
where $\mathbf{z}_{\cdot,i} \in \mathbb{R}^{n}$ is the vector of constriction degrees measured at the $i$th constriction location; 
%
$\mathbf{\hat{z}}_{\cdot,i} \in \mathbb{R}^{n}$ is the corresponding vector of estimated constriction degrees at the $i$th constriction location; and 
%
entry $c_{k,k}$ of the diagonal weight matrix $\mathbf{C}(\mathbf{w})$ is defined by the tri-cubic kernel function $K$ centered in the neighborhood of $\mathbf{w}$.
%
\begin{align}\label{eq:gaussiankernel}
c_{k,k} (\mathbf{w})
&=
K(\mathbf{w},\mathbf{w}_{k,\cdot}) \\
&= 
\begin{cases}
\left( 1 - \left( \frac{\lvert\lvert \mathbf{w}_{k,\cdot} - \mathbf{w} \rvert\rvert}{h} \right)^3 \right)^3 & \lvert\lvert \mathbf{w}_{k,\cdot} - \mathbf{w} \rvert\rvert < h \\
0 & \text{otherwise}
\end{cases} 
\end{align}
%
The scale $h$ sets the size of the spherical neighborhood around $\mathbf{w}$ within which the forward map is computed. 
%
Points close to $\mathbf{w}$ contribute more to the forward map estimator than points at the edge of the neighborhood.
%
The neighborhood contains $\lfloor fn \rfloor$ points, where $f$ is a free parameter. 
%
The expression for the estimator of the forward kinematic map in the neighborhood of $\mathbf{w}$ is the following. 
\begin{equation}
\mathbf{\hat{G}} (\mathbf{w})
=
\left( \mathbf{W}^\intercal \mathbf{C}(\mathbf{w}) \mathbf{W} \right)^{-1} \mathbf{X}^\intercal \mathbf{C}(\mathbf{w}) \mathbf{Z}
\end{equation}











\subsection{Cross-validation of the direct and differential kinematics}

We evaluated the direct and differential kinematics with the root mean squared error (RMSE).
%
Error of the direct and differential kinematics is important parameter, as the forward kinematics underlies the technical performance of the articulator synergy biomarker.
%
The RMSE was computed using \num{10}-fold cross-validation. 
%
In each fold, the cross-validation assigned 90\% of the data-point indices to the training set $\mathcal{S}$ and 10\% to the test set $\mathcal{T}$.
%
No two folds had overlapping test sets. 


The $\mathrm{RMSE}_{\mathbf{G},k}$ for the forward kinematic map $\mathbf{G}$ at the $k$th phonetic place of articulation reflects deviation in the estimated constriction degrees $\mathbf{\hat{z}}_{\cdot,k} = \mathbf{\hat{G}}\left( \mathbf{w}_{j,\cdot} \right) \left[ \begin{array}{c c} 1 & \mathbf{w}_{j,\cdot}^\intercal \end{array} \right]^\intercal$ from the observed constriction degrees $\mathbf{z}_{\cdot,k}$.
%
\begin{equation}
\mathrm{RMSE}_{\mathbf{G},k} = \sqrt{ \sum_{j\in\mathcal{T}} \left( z_{j,k} - \hat{z}_{j,k} \right)^2 \bigg/ \left( \lvert \mathcal{T} \rvert - 1 \right) }
\end{equation}
The $\mathrm{RMSE}_{\mathbf{J},k}$ for the jacobian matrix $\mathbf{J}$ of the forward kinematic map at the $k$th phonetic place of articulation reflects deviation in the estimated finite differences in constriction degrees $\mathbf{\Delta \hat{z}}_{\cdot,k} = \mathbf{\hat{J}}\left( \mathbf{w}_{j,\cdot} \right) \Delta \mathbf{w}_{j,\cdot}$ from the observed finite differences in constriction degrees $\Delta \mathbf{z}_{\cdot,k}$.
%
\begin{equation}
\mathrm{RMSE}_{\mathbf{J},k} = \sqrt{ \sum_{j\in\mathcal{T}} \left( \Delta z_{j,k} - \Delta \hat{z}_{j,k} \right)^2 \bigg/ \left( \lvert \mathcal{T} \rvert - 1 \right) },
\end{equation}
%
where the finite difference $\Delta z_{j,k}$ is obtained by the central difference formula.
%
\begin{equation}
\Delta z_{j,k} = \left( z_{j+1,k} - z_{j-1,k} \right) / 2
\end{equation}


The evaluation was performed for scale parameter $f$ in the range of \num{0.2} to \num{0.9} (i.e., for neighborhoods containing \SI{20}{\percent} to \SI{90}{\percent} of training data-points). 
%
For a given $f$-value and phonetic place of articulation, the 10-fold cross-validation produced \num{10} $\mathrm{RMSE}_{\mathbf{G},k}$ values and \num{10} $\mathrm{RMSE}_{\mathbf{J},k}$ values.
%
The reported $\mathrm{RMSE}_{\mathbf{G},k}$ and $\mathrm{RMSE}_{\mathbf{J},k}$ values are the medians of these \num{10} values.



The RMSE of the forward kinematic map was smaller than the 2.4 mm in-plane spatial resolution of the real-time MRI pulse sequence when \SIrange{20}{90}{\percent} training data-points were in the neighborhood (i.e., for all $f\in \left[ 0.2, 0.9\right]$; see Fig.~\ref{fig:cverrors}). With the exception of the pharyngeal place of articulation, the RMSE was smaller than the standard deviation of the observed constriction degrees for all participants and for all neighborhood sizes.


The RMSE of the jacobian matrix was smaller than the 2.4 in-plane spatial resolution of the real-time MRI pulse sequence when \SIrange{20}{90}{\percent} training data-points were in the neighborhood (i.e., for all $f\in \left[ 0.2, 0.9\right]$; see Fig.~\ref{fig:cverrors}).
%
The RMSE was smaller than the standard deviations of observed frame-to-frame finite differences in constriction degree for all participants and for all neighborhood sizes. 
%
However, for many participants, the RMSE approached the standard deviation of the frame-to-frame finite differences in constriction degree, especially for the velar and pharyngeal places of articulation.
%
The reason for this is that RMSE for the jacobian matrix was small, but the frame-to-frame differences in constriction degrees varied over a small range to begin with. 
% 
Whether the RMSE of the jacobian matrix was small enough to reliably quantify speech behavior was quantified through the precision of the articulator synergy biomarker. 
% 
The next section introduces and evaluates the articulator synergy biomarker. 

\begin{figure*}
\raggedright

\includegraphics[width=\linewidth]{ErrorFigure.pdf}

\caption{(color online) 
{\bf (a)} Root mean squared error (RMSE) of the forward kinematic map estimator of constriction degrees and {\bf (b)} RMSE of the jacobian matrix estimator of frame-to-frame finite differences in constriction degrees. 
Data-points are the median RMSE computed over all 10 folds of cross-validation.
Lines connect the RMSE values of a single participant at different neighborhood sizes ($X$-axis).
Neighborhood size is given as percentage of training data-points.
The standard deviations of observed (frame-to-frame finite differences in) constriction degrees are indicated as tick marks on the right $Y$-axis for each participant whenever the standard deviations are small enough to fit within the $Y$-axis limits.}
\label{fig:cverrors}
\end{figure*}




\section{Articulator synergy biomarkers}
\label{sec:articulator synergy biomarker}

\subsection{Biomarker definition}

In the neighborhood of $\mathbf{w}(t)$, the jacobian $\mathbf{J}(\mathbf{w}(t))$ of the forward kinematic map provided the following relation between change $\mathbf{\dot{z}}(t)$ in constriction degrees and change $\mathbf{\dot{w}}(t)$ in articulator shape and position.
%
\begin{align}
\label{eq:biomarkerderivationcontinuoustime}
\begin{split}
\int_{0}^{T} \mathbf{\dot{z}}(t) \, \mathrm{d}t
	&= \int_{0}^{T} \mathbf{J}\left( \mathbf{w}(t) \right) \mathbf{\dot{w}}(t) \, \mathrm{d}t \\
    &= \sum_{k=1}^q \int_{0}^{T} \mathbf{J}\left( \mathbf{w}(t) \right) \mathbf{P}_k \mathbf{\dot{w}}(t) \, \mathrm{d}t
\end{split}
\end{align}
%
Time $0$ is the temporal onset of a constriction, time $T$ is the temporal offset of a constriction, and the $7\times 7$ diagonal projection matrix $\mathbf{P}_k$ has the $(k,k)$-entry equal to unity and all other entries equal to zero, breaking the integral down into the contributions of each factor score.
Term $k$ of the outer summation is the theoretical contribution of factor $\mathbf{f}_{\cdot,k}$ to elapsed change in constriction degrees $\mathbf{z}$ during a constriction. 

Since real-time MRI provides a discretized sequence of images, the constriction degrees $\mathbf{z}$ and factor scores $\mathbf{w}$ are discrete-time signals. 
%
The discrete-time version of Equation~\ref{eq:biomarkerderivationcontinuoustime} is the following.
%
\begin{align}
\begin{split}
\sum_{j=0}^{N} \Delta \mathbf{z}_{j,\cdot}
	&= \sum_{j=0}^{N} \mathbf{J}\left( \mathbf{w}_{j,\cdot} \right) \Delta \mathbf{w}_{j,\cdot} \\
    &= \sum_{k=1}^q \sum_{j=0}^{N} \mathbf{J}\left( \mathbf{w}_{j,\cdot} \right) \mathbf{P}_k \Delta \mathbf{w}_{j,\cdot}
\end{split}
\end{align}
%
Sample $0$ is the temporal onset of a constriction, and sample $N$ is the temporal offset of a constriction.
%
As in the continuous-time Equation~\ref{eq:biomarkerderivationcontinuoustime}, term $k$ of the outer summation is the contribution of factor $\mathbf{f}_{\cdot,k}$ to elapsed change in constriction degrees $\mathbf{z}$ during a constriction.



The discrete-time signal $\lambda_{\mathcal{U},\ell}[n]$ is the elapsed contribution of the articulator whose factor indices are in the set $\mathcal{U}$ to narrowing or widening constriction degree $z_\ell$. 
\begin{align}
\lambda_{\mathcal{U},\ell} \left[ n \right]
	= \sum_{k\in \mathcal{U}} \sum_{j=0}^{n} \mathbf{j}_{\ell,\cdot} \mathbf{P}_k \Delta \mathbf{w}_{j,\cdot}
\end{align}
The set $\mathcal{U}$ contained the indices of factors corresponding to the target articulator. For example, when the numbers of factors were $q_\text{jaw} = 1$, $q_\text{tongue} = 4$, $q_\text{lip} = 2$, and $q_\text{velum} = 1$, then  $\mathcal{U}=\{1\}$ was the jaw; $\mathcal{U}=\{2,3,4,5\}$ was the tongue; $\mathcal{U}=\{6,7\}$ was the lips; and $\mathcal{U}=\{8\}$ was the velum.
%
As $n$ increases from the onset $0$ of a constriction to the offset $N$ of the subsequent release, the signal $\lambda_{\mathcal{U},\ell}[n]$ dips to a minimum and then rises back up. 



Let $\mathcal{J}$ be the set of jaw factor indices and let $\mathcal{E}$ be the set of lip or tongue factor indices. Specifically, the set $\mathcal{E}$ contains lip factor indices for the bilabial place of articulation and contains tongue factor indices for the alveolar, palatal, velar, and pharyngeal places of articulation. 
%
We defined the articulator synergy biomarker $\nu_\ell$ for place of articulation $\ell$ as the range of $\lambda_{\mathcal{J},\ell} [ n ]$ divided by the range of $\lambda_{\mathcal{J},\ell} [ n ] + \lambda_{\mathcal{E},\ell} [ n ]$
over all samples $n \in \{0, 1, 2, \ldots, N\}$.
%
Range was computed as the difference between the 90th percentile $P_{90}$ and 10th percentile $P_{10}$. 
%
Thus, the articulator synergy biomarker $\nu_\ell$ was the following ratio.
\begin{equation}
\nu_\ell
=
\frac{P_{90}\left( \lambda_{\mathcal{J},\ell} [n] \right) - P_{10}\left( \lambda_{\mathcal{J},\ell} [n] \right)}
{P_{90}\left( \lambda_{\mathcal{J},\ell} [n] + \lambda_{\mathcal{E},\ell} [n] \right) - P_{10}\left( \lambda_{\mathcal{J},\ell} [n] + \lambda_{\mathcal{E},\ell} [n]\right)}
\end{equation}
%
The articulator synergy biomarker $\nu_\ell$ is the percent contribution of the jaw to narrowing and widening the vocal tract for a constriction. 
%
The quantity $1-\nu_\ell$ is the percent contribution of the lips (for the bilabial place) or the tongue (for the alveolar, palatal, velar, and pharyngeal places) to a constriction. 


\begin{figure}

\includegraphics[width=0.8\linewidth]{RealtimeMRIFigure.pdf}

\caption{(color online) Biomarker extraction from real-time magnetic resonance imaging of a constriction at the palatal place in the transition from vowel \textipa{[a]} to vowel \textipa{[j]} in the sequence \textipa{[aja]}. Reconstructed frame rate was downsampled for presentation. 
\textbf{a.} Constriction degree at the palatal place is the length of the green line connecting the tongue and hard palate (magenta). 
\textbf{b.} Jaw contribution to constriction degree change.
\textbf{c.} Tongue contribution to constriction degree change.
\textbf{d.} Total constriction degree change is the sum of jaw and tongue contributions.
\textbf{e.} The articulator synergy biomarker resolves the elapsed change in constriction degree into jaw and tongue contributions.}
\label{fig:realtimeMRI}%
\end{figure}





\subsection{Biomarker precision}
\label{subsec:repeatability}

Precision is the agreement between replicate measurements of a vocal tract constriction by the same participant at the same phonetic place of articulation~\citep{kessler2015emerging,sullivan2015metrology}. 
%
The same-day test-retest repeatability experiment evaluated the repeatability of the articulator synergy biomarker under variable conditions of MRI operator variability, image analysis variability, and short-term physiological variability. 
%
MRI operator variability included subject positioning within the scanner bore and scan plane localization. 
%
Image analysis variability included variability in the manual step of time-point annotation and in the manual initialization of the segmentation algorithm. 
%
Short-term physiological variability included same-day scan-to-scan variability in speech motor control. 
%
Precision is an important parameter, as it establishes a limit on effect size and group differences that the method can resolve.

Agreement between Scan~\num{1} and Scan~\num{2} was quantified using the intraclass correlation coefficient (ICC). The ICC is a quantitative measure of test-retest repeatability for articulator synergy biomarkers. 
The ICC is the ratio of inter-participant variability to total variability. The greater the inter-participant variability compared to total variability, the greater the reliability, because random error is smaller relative to the variance of the articulator synergy biomarker between experiment participants. 
On the basis of a recent review~\citep{lebreton2007answers}, ICC values were categorized as poor (\num{0.00} to \num{0.30}), weak (\num{0.31} to \num{0.50}), moderate (\num{0.51} to \num{0.70}), strong (\num{0.71} to \num{0.90}), and very strong (\num{0.91} to \num{1.00}).
%
The ICC was computed using a linear mixed effects model fitted with the package lme4~\citep{bates2015fitting} in R~\citep{r2017language}. Consider the sample of $n=8$ participants, each with $k=20$ repeated measurements of articulator synergy (\num{10} from Scan~\num{1}, \num{10} from Scan~\num{2}). The articulator synergy biomarker $\nu _{ij}$ for replicate measurement $j$ and participant $i$ was 
%
\begin{equation}
\nu _{ij} = m + p_i + e_{ij},
\end{equation}
%
where $m$ was the group mean, $p_i$ was the random intercept for participant $i$, and $e_{ij}$ was the intra-participant error. 
%
The random effects $p_i$ and $e_{ij}$ were independently and identically distributed with mean 0 and the inter-participant variance $\sigma_p^2$ and intra-participant variance $\sigma_e^2$ to be estimated from the data using the restricted maximum likelihood procedure. 
% 
The ICC is the proportion of variance in the articulator synergy biomarker value due to biological variation among participants, compared to the total variance of the articulator synergy biomarker.
%
\begin{equation}
\text{ICC}(\nu) = \frac{\hat{\sigma}_p^2}{\hat{\sigma}_p^2 + \hat{\sigma}_e^2}
\end{equation}
%
If the ICC is close to \num{1}, variance in the articulator synergy biomarker mostly reflects variability among participants. 
%
If the ICC is close to \num{0}, then variability among participants accounts for little of the total variance in the articulator synergy biomarker. 

The repeatability of the articulator synergy biomarker was evaluated for different numbers of jaw, tongue, and lip factors (Fig.~\ref{fig:icc_all}). 
%
The images of participant F3 were excluded from the repeatability analysis due to poor image quality in the second scan. 



Repeatability of the articulator synergy biomarkers ranged from poor to strong over the wide range of factor analysis parameterizations tested. 
%
% bilabial place
%
The bilabial place had
moderate  to  strong
repeatability
(range: \numrange{0.6}{0.71},
median: \num{0.68}). 
%
% alveolar place
% 
The alveolar place had
poor  to  moderate
repeatability
(range: \numrange{0.21}{0.52},
median: \num{0.36}). 
%
% palatal place
%
The palatal place had
moderate
repeatability
(range: \numrange{0.54}{0.65},
median: \num{0.58}). 
%
% velar place
%
The velar place had
weak  to  moderate
repeatability
(range: \numrange{0.31}{0.6},
median: \num{0.44}). 
%
% pharyngeal place
%
The pharyngeal place had 
poor  to  weak
repeatability
(range: \numrange{0.22}{0.48},
median: \num{0.36}). 


\begin{figure*}
\raggedright

\includegraphics[width=\linewidth]{ICCFigure.pdf}

\caption{(color online) Comparison of intraclass correlation coefficients (ICC) for different numbers of jaw factors (color) and for different numbers of tongue and lip factors ($x$-axis) at the bilabial, alveolar, palatal, velar, and pharyngeal places of articulation.}
\label{fig:icc_all}
\end{figure*}





The range of ICC values obtained at the alveolar, velar, and pharyngeal places of articulation included some ICC values in the poor to weak range. 
%
The reason for this may be different for the different places of articulation. 
%
At the velar place of articulation, the total variance of the biomarker is small
(inter-quartile range: \SIrange{6}{13}{\percent}). 
Even if the intra-participant variance is small to begin with, the intra-participant variance makes up a substantial part of the total variance (see the histograms in Fig.~\ref{fig:histograms}). 
%
In contrast, at the alveolar and pharyngeal places of articulation, the total variance of biomarker is large 
(alveolar inter-quartile range: \SIrange{43}{63}{\percent};
pharyngeal inter-quartile range: \SIrange{13}{32}{\percent}),
suggesting that intra-participant variance is substantial (see the histograms in Fig.~\ref{fig:histograms}). 


\begin{figure*}

\includegraphics[width=\linewidth]{HistogramsFigure.pdf}

\caption{\label{fig:histograms}(color online) 
Sample distribution of the jaw contributions to constrictions at the bilabial, alveolar, palatal, velar, and pharyngeal places of articulation. The quantity plotted is the percent of a constriction that was produced by the jaw.
A value of \SI{0}{\percent} indicates that lip or tongue motion produced the entire constriction, whereas a value of \SI{100}{\percent} indicates that jaw motion produced the entire constriction. 
Sample distribution by participant shown with a different color for each participant.}

\end{figure*}




\section{Testing the task-specificity of articulator synergies}
\label{sec:taskspec}

The present study tested the hypothesis that the jaw contributed to anterior constrictions at the bilabial, alveolar, and palatal places but not to posterior constrictions at the velar and pharyngeal places using a linear mixed effects model fitted with the package lme4~\citep{bates2015fitting} in R~\citep{r2017language}. 


Consider the sample of $n=320$ articulator synergy biomarker values (\num{8} participants $\times$ \num{4} places of articulation $\times$ \num{10} repeated measurements of the articulator synergy biomarker from scan \num{1}). Let $y_{i,j,k}$ be the biomarker value for place of articulation $i$, participant $j$, and replicate measurement $k$ from scan $\ell$ (i.e., Scan~\num{1} or Scan~\num{2}). The linear mixed effects model of $y_{i,j,k,\ell}$ was 
%
\begin{equation}
y_{i,j,k,\ell} = m + b_i + p_j + q_{j,k} + r_{j,\ell} + c_\ell + e_{i,j,k,\ell},
\end{equation}
%
where $m$ was the baseline mean, $b_i$ was the fixed effect for place of articulation, $c_\ell$ was the fixed effect for scan number, $p_i$ was the random intercept for participant $i$, $q_{j,k}$ was the by-participant random slope for place of articulation, $r_{j,\ell}$ was the by-participant random slope for scan number, and $e_{i,j,k}$ was the intra-participant error. Multiple comparisons were corrected for using Tukey's range test with the package multcomp~\citep{hothorn2008simultaneous} in R~\citep{r2017language}. This section reports adjusted $p$-values. 

% velar comparisons
%
On average, the percent jaw contribution was 
%
% velar vs. bilabial
%
\SI{17}{\percent} 
less at the velar place compared to the bilabial place
($z=1.9$, 
$p=0.19$),
%
% velar vs. alveolar
%
\SI{40}{\percent} 
less at the velar place compared to the alveolar place
($z=7$, 
$p=\ensuremath{1.6\times 10^{-11}}$), and
%
% velar vs. palatal
%
\SI{19}{\percent} 
less at the velar place compared to the palatal place
($z=2.4$, 
$p=0.06$).




% pharyngeal comparisons
%
On average, the percent jaw contribution was 
%
% pharyngeal vs. bilabial
%
\SI{5.6}{\percent} 
less at the pharyngeal place compared to the bilabial place
($z=0.54$, 
$p=0.96$),
%
% pharyngeal vs. alveolar
%
\SI{28}{\percent} 
less at the pharyngeal place compared to the alveolar place
($z=4$, 
$p=\ensuremath{2.7\times 10^{-4}}$), and
%
% pharyngeal vs. palatal
%
\SI{8.1}{\percent} 
less at the pharyngeal place compared to the palatal place
($z=1.4$, 
$p=0.5$).




We reject the null hypothesis that the articulator synergy biomarker does not differ between velar and alveolar and between pharyngeal and alveolar places of articulation. 
%
We infer that the jaw contributed significantly more to constrictions at the alveolar place compared to constrictions at the velar and pharyngeal places.
%
This indicates that synergies with different places of articulation may differ in terms of inter-articulator coordination.




The sample distribution of the articulator synergy biomarker at the velar place had small dispersion about a distinct peak at \SI{10}{\percent} 
%
(median: \SI{10}{\percent}, 
inter-quartile range: \SIrange{6}{13}{\percent}; see histograms in Fig.~\ref{fig:histograms}).
%
The sample distribution of the articulator synergy biomarker at the alveolar place had a distinct peak at \SI{60}{\percent}
%
(median: \SI{55}{\percent}, 
inter-quartile range: \SIrange{43}{63}{\percent}; see histograms in Fig.~\ref{fig:histograms}).
%
The distinctly peaked sample distributions of the articulator synergy biomarkers at the alveolar and velar places likely contributed to the statistically significant alveolar-velar and alveolar-pharyngeal differences.



In contrast to the alveolar-velar and alveolar-pharyngeal comparisons, the biomarker values at the bilabial and palatal places of articulation did not differ significantly from those at the velar and pharyngeal places of articulation. 
%
The large dispersion of the biomarker sample distribution at the bilabial place
%
(median: \SI{24}{\percent}, 
inter-quartile range: \SIrange{9.4}{49}{\percent}), 
%
and palatal place
%
(median: \SI{26}{\percent}, 
inter-quartile range: \SIrange{12}{45}{\percent})
%
suggests that participants exhibited substantial heterogeneity in biomarker values at these places of articulation.  










\section{Discussion}

The research goals of the present study were (i) to estimate and evaluate the direct and differential kinematics of the vocal tract from MRI, 
%
(ii) to design and evaluate an articulator synergy biomarker based on the estimated kinematics, and
%
(iii) to use the articulator synergy biomarker to test the task-dependence of articulator synergies by determining whether the jaw contributes more for anterior constrictions at the bilabial, alveolar, and palatal places of articulation than for posterior constrictions at the velar and pharyngeal places of articulation.
%
The study demonstrated low error in the estimator of the direct and differential kinematics and consistent estimation of the articulator synergy biomarker in a test-retest repeatability experiment.
%
A comparison of the parameters of a mixed effects model fitted to the biomarker values demonstrated that speakers use the jaw more to produce constrictions at the alveolar place of articulation than to produce constrictions at the velar and pharyngeal places of articulation. 
%
The pattern of jaw, tongue, and lip usage was heterogeneous for the bilabial and palatal places of articulation.
%
The implication of the results is that the articulator synergy biomarker has adequate precision to characterize healthy variability in articulator synergies, and that these synergies vary by speaker and by place of articulation. 
%
The present study supports the idea that different articulator synergies have different patterns of inter-articulator coordination. 
If synergies organize the articulators on a temporary basis for achieving motor goals such as vocal tract constrictions, as proposed in theories of motor control~\citep{turvey1977preliminaries, saltzman1987skilled} and in theories of phonological organization~\citep{browman1986towards, browman1989articulatory}, then the pattern of inter-articulator coordination varies over time as the vocal tract deploys different synergies. 
The articulator synergy biomarker provides the means to characterize this time-varying pattern of inter-articulator coordination on the basis of a computational model of the direct and differential kinematics of the vocal tract. 

The algorithm for computing the articulator synergy biomarker involves a computational model of the forward kinematic map~\citep{lammert2013statistical}, based on the Task Dynamics model of speech production~\citep{saltzman1989dynamical}. In contrast to articulator synergy biomarkers that use a priori physiological models to compute biomarker values~\citep[e.g., articulator synergy biomarkers of kinetic parameters from dynamic contrast-enhanced MRI,][]{aerts2008system},  the articulator synergy biomarker algorithm fits the forward kinematic map to the real-time MRI data and uses the fitted model, not an a priori model, to compute the articulator synergy biomarker. Thus, the proposed algorithm took into account by-participant variability in the forward kinematic map by estimating the forward kinematic map for each participant. 

The proliferation of vocal tract imaging databases~\citep{narayanan2014real,sorensen2017database} and the increasingly complex computational methods for studying the morphological~\citep{lammert2013morphological} and functional~\citep{dawson2016methods} complexities captured therein underscore the importance of evaluating the technical performance of quantitative imaging biomarkers of speech. The model-based approach to computing articulator synergy biomarkers allowed the present study to not only evaluate biomarker performance, but to attribute error to specific components of the model, such as the estimator of the jacobian matrix. This provided an understanding of where error is introduced. This approach revealed that poor articulator synergy biomarker performance for the pharyngeal place of articulation was due to error in the estimator of the corresponding row of the jacobian matrix of the forward kinematic map. The error approached the magnitude of the standard deviation of the frame-to-frame finite differences in constriction degrees observed at the pharyngeal place of articulation. Although a strength of MRI over competing imaging modalities such as ultrasound is that MRI allows visualization of the pharynx, this region has a complicated geometry which makes reliably processing images of this region difficult. Specifically, the shape and position of the epiglottis depends on the anterior-posterior position of the tongue, as well as subject-specific anatomical and physiological properties. With the tongue root advanced, air may intervene between the tongue root and epiglottis, whereas no such air-tissue interface exists with the tongue root retracted. This leads to drastic differences in the image depending on the state of the tongue. The present study demonstrated a need for reliable quantification of hypopharyngeal structures such as the epiglottis. Developing reliable segmentation of the hypopharynx may improve the technical performance of the the pharyngeal synergy biomarkers.

Our previous study showed that real-time MRI could be used to obtain articulator synergy biomarkers of the range of motion at the phonetic places of articulation, but showed poor precision for articulator synergy biomarkers of articulator velocities~\citep{toger2017test}. Despite the importance of velocity information for obtaining articulator synergy biomarkers, the technical performance of the articulator synergy biomarker nevertheless represents an improvement over articulator synergy biomarkers of articulator velocities. This improvement is likely due to the algorithm not using raw articulator velocities, but rather transforming and integrating multiple articulator velocities over time. 






\section{Conclusions}

The present study estimated and evaluated the direct and differential kinematics of the vocal tract from MRI, 
%
designed and evaluated an articulator synergy biomarker based on the estimated kinematics, and
%
used the articulator synergy biomarker to test the task-dependence of articulator synergies.
%
The study demonstrated that the articulator synergy biomarker has adequate precision to characterize healthy variability in articulator synergies, and that these synergies vary by speaker and by place of articulation. 
%
The implication of the results is that the articulator synergy biomarker has adequate precision to characterize healthy variability in articulator synergies, and that articulator synergies are task-dependent in that they have different patterns of inter-articulator coordination depending on their place of articulation. 



The following are two potential threads of research that build on the results of the present study.
First, the articulator synergy biomarker was computed using the forward kinematics of the vocal tract, which relates articulator motion to the resulting changes in constriction degrees at the phonetic places of articulation. Future studies may investigate the inverse kinematics, which relates changes in constriction degrees to the optimal articulator movements for producing them. 
%
Second, the present study showed that the articulator synergy biomarker varied depending on the place of articulation. Future studies may evaluate whether the articulator synergy biomarker further depends on sociolinguistic factors, sex, anatomy, and age, especially during childhood.

\section{Reproducibility and replication}

The scripts required to reproduce the present study are available as supplementary material.\footnote{The supplementary material required to reproduce the present study is available at [URL will be inserted by AIP]}
%
The MRI data-set is available at \url{http://sail.usc.edu/span/test-retest/} for free use by the research community~\citep[see][]{toger2017test}.
%
A replication of the present study was performed using the USC Speech and Vocal Tract Morphology MRI Database~\citep{sorensen2017database}.\footnote{The results of the replication study are available at [URL will be inserted by AIP].}

\section{Acknowledgments} 

The authors acknowledge funding through NIH grant R01DC007124, NIH grant T32DC009975, and NSF grant 1514544. The content of this paper is solely the responsibility of the authors and does not necessarily represent the official views of the NIH or NSF.

\bibliography{mybib.bib}

\end{document}
