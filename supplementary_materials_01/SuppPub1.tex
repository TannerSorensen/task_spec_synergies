\documentclass[reprint]{JASAnew}\usepackage[]{graphicx}\usepackage[]{color}
%% maxwidth is the original width if it is less than linewidth
%% otherwise use linewidth (to make sure the graphics do not exceed the margin)
\makeatletter
\def\maxwidth{ %
  \ifdim\Gin@nat@width>\linewidth
    \linewidth
  \else
    \Gin@nat@width
  \fi
}
\makeatother

\definecolor{fgcolor}{rgb}{0.345, 0.345, 0.345}
\newcommand{\hlnum}[1]{\textcolor[rgb]{0.686,0.059,0.569}{#1}}%
\newcommand{\hlstr}[1]{\textcolor[rgb]{0.192,0.494,0.8}{#1}}%
\newcommand{\hlcom}[1]{\textcolor[rgb]{0.678,0.584,0.686}{\textit{#1}}}%
\newcommand{\hlopt}[1]{\textcolor[rgb]{0,0,0}{#1}}%
\newcommand{\hlstd}[1]{\textcolor[rgb]{0.345,0.345,0.345}{#1}}%
\newcommand{\hlkwa}[1]{\textcolor[rgb]{0.161,0.373,0.58}{\textbf{#1}}}%
\newcommand{\hlkwb}[1]{\textcolor[rgb]{0.69,0.353,0.396}{#1}}%
\newcommand{\hlkwc}[1]{\textcolor[rgb]{0.333,0.667,0.333}{#1}}%
\newcommand{\hlkwd}[1]{\textcolor[rgb]{0.737,0.353,0.396}{\textbf{#1}}}%
\let\hlipl\hlkwb

\usepackage{framed}
\makeatletter
\newenvironment{kframe}{%
 \def\at@end@of@kframe{}%
 \ifinner\ifhmode%
  \def\at@end@of@kframe{\end{minipage}}%
  \begin{minipage}{\columnwidth}%
 \fi\fi%
 \def\FrameCommand##1{\hskip\@totalleftmargin \hskip-\fboxsep
 \colorbox{shadecolor}{##1}\hskip-\fboxsep
     % There is no \\@totalrightmargin, so:
     \hskip-\linewidth \hskip-\@totalleftmargin \hskip\columnwidth}%
 \MakeFramed {\advance\hsize-\width
   \@totalleftmargin\z@ \linewidth\hsize
   \@setminipage}}%
 {\par\unskip\endMakeFramed%
 \at@end@of@kframe}
\makeatother

\definecolor{shadecolor}{rgb}{.97, .97, .97}
\definecolor{messagecolor}{rgb}{0, 0, 0}
\definecolor{warningcolor}{rgb}{1, 0, 1}
\definecolor{errorcolor}{rgb}{1, 0, 0}
\newenvironment{knitrout}{}{} % an empty environment to be redefined in TeX

\usepackage{alltt}

\graphicspath{{./graphics/}}

% symbols
\usepackage[utf8]{inputenc}
\usepackage{siunitx}
\usepackage{tipa}
\usepackage{amsmath,mathtools,amssymb}
\DeclareSIUnit\frame{frames}

% URLs
\usepackage{url}

% table formatting
\usepackage{multirow,array,longtable}
\newcommand\Tstrut{\rule{0pt}{2.6ex}}         % = `top' strut
\newcommand\Bstrut{\rule[-0.9ex]{0pt}{0pt}}   % = `bottom' strut
\newcolumntype{L}[1]{>{\raggedright\let\newline\\\arraybackslash\hspace{0pt}}m{#1}}
\newcolumntype{C}[1]{>{\centering\let\newline\\\arraybackslash\hspace{0pt}}m{#1}}
\newcolumntype{R}[1]{>{\raggedleft\let\newline\\\arraybackslash\hspace{0pt}}m{#1}}
\usepackage{stackengine}
\IfFileExists{upquote.sty}{\usepackage{upquote}}{}
\begin{document}

\section{Introduction}

This document reports the results of a partial replication of the study ``Task dependence of articulator synergies''. This is a replication of the error analysis of the direct and differential kinematics of the vocal tract and of the finding that articulator synergies are task dependent. The precision is not replicated because the USC Speech and Vocal Tract Morphology MRI Database was not acquired in a test-retest experiment. Section~\ref{sec:database} describes the USC Speech and Vocal Tract Morphology MRI Database. Section~\ref{sec:crossvalidation} replicates the direct and differential kinematics of the vocal tract. Section~\ref{sec:taskdependence} replicates the finding that articulator synergies are task dependent. 

\section{USC Speech and Vocal Tract Morphology MRI Database}
\label{sec:database}


The volunteers in the USC Speech and Vocal Tract Morphology MRI Database~\citep{sorensen2017database} were healthy male and female speakers of American English.
%
The database included seventeen (8 male, 9 female) speakers of American English. The parents of each participant were native speakers of American English. None of the participants reported abnormal hearing or speech pathology. See Table~\ref{tab:subj1} for participant age, gender, and state of origin. 

\begin{longtable}{l l l l}
\hline
ID & age & gender & state of origin \Tstrut \Bstrut \\
\hline
F1	& 25 & F & California  \Tstrut \\ %San Clemente, CA 
F2	& 25 & F & New York \\ %Commack, NY
F3	& 26 & F & 	California \\ %Brawley, CA
F4	& 25 & F & 	Washington, D.C. \\ %Washington, DC
F5	& 28 & F & 	South Carolina \\ %West Columbia, SC
F6	& 31 & F & Hawaii \\ %Honolulu, HI
F7	& 64 & F & Minnesota \\ %St. Paul, MN
F8	& 26 & F & 	Texas \\ %Houston, TX
F9	& 22 & F & Rhode Island \\ %Providence, RI
M1 	& 33 & M & Wisconsin \\%Eau Claire, WI \Tstrut \\
M2	& 27 & M & Virginia \\ %Richmond, VA\\
M3	& 28 & M & Wisconsin \\ %Madison, WI\\
M4	& 20 & M & California \\ %West Covina, CA\\
M5	& 38 & M & Washington, D.C. \\ %Washington, DC\\	
M6	& 24 & M &	New Jersey \\ %Newark, NJ\\
M7	& 33 & M & Texas \\ %San Antonio,TX\\
M8 	& 26 & M & Iowa \Bstrut \\ %Iowa City, IA\\
\hline
& \shortstack[l]{\Tstrut median: 25 \\ range: 20--64} 
& \shortstack[l]{\Tstrut 8 male \\ 9 female}
& \Tstrut \Bstrut \\
\hline 
\caption{Participant characteristics of the USC Speech and Vocal Tract Morphology MRI Database}
\label{tab:subj1}
\end{longtable}



Each speaker participated in one session. The study personnel explained the nature of the experiment and the protocol to the participant before each scan. The participant lay on the scanner table in a supine position. The head was fixed in place by foam pads inserted between the temple and the receiver coil on the left and right sides of the head. The participant read visual stimuli from a back-projection screen from inside the scanner bore without moving the head. The speech corpus included real-time MRI videos of the isolated vowel-consonant-vowel utterances [apa], [ata], [aka], [aja]. The participant produced each vowel-consonant-vowel utterance three times. After completing the session, the speaker was paid for their participation in the study. The USC Institutional Review Board approved the data collection procedures. The MRI data-set is available at \url{http://sail.usc.edu/span/morphdb/} for free use by the research community.


Data were acquired on a Signa Excite HD \SI{1.5}{\tesla} scanner (General Electric Healthcare, Waukesha WI) with gradients capable of \SI[per-mode=symbol]{40}{\milli\tesla\per\meter} amplitude and \SI[per-mode=repeated-symbol]{150}{\milli\tesla\per\meter\per\milli\second} slew rate. 
%
A body coil was used for radio frequency (RF) signal transmission. A custom upper airway receiver coil array was used for RF signal reception. This 4-channel array included two coil elements anterior to the head and neck and two coil elements posterior to the head and neck. Only the two anterior coils were used for data acquisition because the posterior coils were shown to result in aliasing artifacts. 
%
The real-time MRI pulse sequence parameters were the following: 
%
\SI{200 x 200}{\milli\meter} field of view, 
\SI{2.9 x 2.9}{\milli\meter} reconstructed in-plane spatial resolution, 
\SI{5}{\milli\meter} slice thickness,
\SI{6.164}{\milli\second} TR,%
\SI{3.6}{\milli\second} TE,
\SI{15}{\degree} flip angle,%
\num{13} spiral interleaves for full sampling.
%
The scan plane was manually aligned to the head. 
%
In reconstruction, a sliding window technique was used to allow for view sharing and thus to increase frame rate~\citep{kim2011flexible,narayanan2004approach}. The TR-increment for view sharing was \num{7}, which resulted in the generation of an MRI video with frame rate $1/(7 \times \text{TR}) = 1/(7 \times \text{\SI{6.164}{\milli\second}}) = \text{\SI[per-mode=symbol]{23.18}{\frame\per\second}}$. 
Localization of the midsagittal scan plane was performed using RTHawk (HeartVista, Inc., Los Altos, CA), a custom real-time imaging platform~\citep{santos2004flexible}. 







\section{Cross-validation of the direct and differential kinematics}
\label{sec:crossvalidation}

For the forward kinematic map, the median error and the 90\textsuperscript{th} percentile of the error was smaller than the \SI{2.9}{\milli\meter} in-plane spatial resolution of the real-time MRI pulse sequence when \SIrange{20}{95}{\percent} of training data-points were in the neighborhood (i.e., for all $f\in \left[ 0.2, 0.95\right]$; see Fig.~\ref{fig:cverrors}). 

For the jacobian matrix, the median error was smaller than the \SI{2.9}{\milli\meter} in-plane spatial resolution of the real-time MRI pulse sequence when \SIrange{20}{95}{\percent} of training data-points were in the neighborhood (i.e., for all $f\in \left[ 0.2, 0.95\right]$; see Fig.~\ref{fig:cverrors}). However, when very few training data-points were in the neighborhood (i.e., for $f=0.2$), the 90\textsuperscript{th} percentile of the error distribution exceeded the \SI{2.9}{\milli\meter} in-plane spatial resolution. 

\begin{figure*}
\raggedright
\includegraphics[width=\linewidth]{SuppPub1_ErrorFigure.pdf}
\caption{
{\bf (a)} Median error (line) and 10\textsuperscript{th}-90\textsuperscript{th} percentile error range (shaded) of the forward kinematic map estimator of constriction task variables. {\bf (b)} Median error (line) and 10\textsuperscript{th}-90\textsuperscript{th} percentile error range (shaded) of the jacobian matrix estimator of frame-to-frame finite differences in constriction task variables. 
Data-points are the errors computed over all 10 folds of cross-validation.
Neighborhood size is given as percentage of training data-points.
The standard deviation of observed (frame-to-frame finite differences in) constriction task variables is indicated as a dashed line whenever the standard deviation is small enough to fit within the $y$-axis limits.}
\label{fig:cverrors}
\end{figure*}















\section{Testing the task-specificity of articulator synergies}
\label{sec:taskdependence}

Table~\ref{tab:stat_results} reports the results of the statistical tests. Compare to Table~II of the primary study. The results of the replication study agree with those of the main study, except for the palatal approximation-pharyngeal approximation contrast and pharyngeal approximation-velar stop contrast. In the primary study, these two contrasts were significant. In the replication, they are not.

\begin{table*}
\centering
\begin{tabular}{l l l l}
contrast & estimate (\%) & $z$ & $p$ \\
\hline
%
% bilabial vs. coronal
%
bilabial stop-coronal stop &
\SI{-38} &
-6.1 &
\ensuremath{4.7\times 10^{-9}} \\
%
% bilabial vs. palatal
%
bilabial stop-palatal approximation &
\SI{-19} &
-2.3 &
0.14 \\
%
% bilabial vs. velar
%
bilabial stop-velar stop &
\SI{3.8} &
0.57 &
0.97 \\
%
% bilabial vs. pharyngeal
%
bilabial stop-pharyngeal approximation &
\SI{-9.3} &
-1.2 &
0.74 \\
%
% coronal vs. palatal
%
coronal stop-palatal approximation &
\SI{19} &
3.1 &
0.015 \\
%
% coronal vs. velar
%
coronal stop-velar stop &
\SI{42} &
6 &
\ensuremath{1.1\times 10^{-8}} \\
%
% coronal vs. pharyngeal
%
coronal stop-pharyngeal approximation &
\SI{29} &
4.3 &
\ensuremath{1.4\times 10^{-4}} \\
%
% palatal vs. velar
%
palatal approximation-velar stop &
\SI{23} &
4.3 &
\ensuremath{2\times 10^{-4}} \\
%
% palatal vs. pharyngeal
%
palatal approximation-pharyngeal approximation &
\SI{10} &
2.7 &
0.053 \\
%
% pharyngeal vs. velar
%
pharyngeal approximation-velar stop &
\SI{13} &
2.4 &
0.093 \\
\hline
\end{tabular}
\caption{Results for statistical tests of the null hypothesis that the contrast is zero. Rows indicate separate tests. $p$-values corrected for multiple comparisons with Tukey's range test (adjusted $p$-values reported).}
\label{tab:stat_results}
\end{table*}

\begin{figure*}

\includegraphics[width=\textwidth]{SuppPub1_HistogramFigure.pdf}

\caption{\label{fig:histograms}
Sample distribution of the articulator synergy biomarker at the bilabial, alveolar, palatal, velar, and pharyngeal places of articulation. The biomarker indicates the percent of a constriction that was produced by the jaw.
A value of \SI{0}{\percent} indicates that lip or tongue motion produced the entire constriction, whereas a value of \SI{100}{\percent} indicates that jaw motion produced the entire constriction. 
Sample distribution by participant shown with a different color for each participant.}

\end{figure*}


\bibliography{mybib.bib}


\end{document}
